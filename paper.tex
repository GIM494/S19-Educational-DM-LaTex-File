\documentclass[12pt]{article}
\usepackage{amsmath}%
\usepackage{amssymb}%
\usepackage{multicol}
\usepackage{graphicx,epsfig}%
\usepackage[margin=1in]{geometry}
\usepackage{rotating}%
\usepackage{url}
\usepackage[backend=bibtex, citestyle=ieee]{biblatex}
\bibliography{paper}
\usepackage{epsfig}%
\usepackage{epstopdf}%
\usepackage{varwidth}
\usepackage{lscape}%
\usepackage{color}
\usepackage[hang,flushmargin]{footmisc} 
\pdfminorversion 3
\usepackage{pbox}

\newcommand{\hilt}[1]{\colorbox{green}{#1}}
\usepackage{setspace}
\usepackage{adjustbox}
\def\x{\mathbf{x}}

\newcommand\blfootnote[1]{%
  \begingroup
  \renewcommand\thefootnote{}\footnote{#1}%
  \addtocounter{footnote}{-1}%
  \endgroup
}

\newcommand{\tcr}{\textcolor{red}}
\newcommand{\tcb}{\textcolor{blue}}
\newcommand\tab[1][1cm]{\hspace*{#1}}

\DeclareMathOperator*{\argmin}{argmin} 
\renewcommand{\baselinestretch}{1.5}
\usepackage{tabularx}

\usepackage{bibentry}

\title{S19: REU Research Document}
\author{Emily Slaughter, Gennie Mansi }

\begin{document}

\maketitle

\clearpage

\tableofcontents

\clearpage

%%%%%%%%%%%%%%%%%%%
\section{Abstract} \label{abstract}



\clearpage
%%%%%%%%%%%%%%%%%%%
\section{Introduction} \label{intro}

Proper documentation forms an integral part of developing and maintaining software, especially given the increasing size of code bases and the prevalence of temporally and spatially dispersed teams. As code is written and the project develops, various design decisions are made, which impact both existing and future code. Documentation of design decisions and the rational behind those decisions is dependent on the developer and is often recorded or dispersed between several disparate locations, including a formal document, comments in the code, and the developers themselves. However, documentation is often not written, read, or updated, causing developers to depend largely on the code and other sources of information rather than explicit, written documentation. 

Ideally, as code is authored or modified, it should follow previous design rules use to author prior code.
The \textit{design rules} for a project is the set of choices about the code's design and how those choices were implemented. Several sources of information, including search history, cursor location, event logs, debugger information, and the code itself can indicate what elements in the code base are most pertinent to the user and are most likely to contain design rules of interest. Some information about design rules may only be recorded in the memories of co-workers, and thus access to this kind of information is also critical for developers trying to understand and use an existing code base. 

Mixed human-AI authoring of code patterns can greatly assist developers to efficiently and accurately develop and maintain a unified set of rules. There are an infinite number of patterns that could be found in a given code base; therefore, the crux of the problem lies in determining which patterns in the given code should be used to train the machine learning algorithm. Hence our focus is on mining design rules from a given code base in any language in order to facilitate active documentation and maintenance of code.

We explore how information from outside of the code base could be used to inform what patterns are identified.We seek to develop a tool that combines that has both automatic and manual rule-authoring capabilities. It should also automatically provide live updates with examples and violations of authored design rules. Ultimately, such a tool will greatly increase the amount of up-to-date documentation available to developers.
 
 
 \clearpage


%%%%%%%%%%%%%%%%%%%
\section{Methods}\label{methods}

\subsection{Problem Description} \label{probDesc}

Given the code for a project and some external information such as cursor information, we wish to derive a set of most likely and relevant design rules. We will then integrate our algorithm for mining these design rules into an interface that allows developers to seamlessly mine and write design rules and to see violations and correct examples of these rules. 

When considering our task, there exist two significant considerations: (a) what kinds of data are available and helpful for suggesting rules, and (b) what kind of algorithm can be utilized with these sources of data in order to make relevant design rule suggestions. We will begin by first addressing different sources of information that are available and most useful for our purpose. Then, we will proceed to discuss our algorithm for design rule mining.


\subsection{Assumptions about the user} \label{assumptions}

Developers can be divided into three categories with respect to their familiarity with a project or code base. First, a developer may be experienced with a given code base. She would have a thorough understanding of the project, its objectives, and its implementation. She is in a position that allows her to author design rules and verify that certain rules have been followed through the project. Second, a developer may be partially knowledgable about a project and its implementation. Such a developer may have a general familiarity with the code and its implementations, but is not able to easily or confidently author design rules or may simply wish to explore possible existing design rules. On her own she is not be able to produce an accurate rule, but might be able to see a rule and provide useful feedback on its correctness or relevancy. Finally, a developer may be completely unfamiliar with a code base. Existing design rules and documents would be extremely helpful in helping her understand the structure of the code. She may also need to frequently consult co-workers about code standards and patterns if there is a significant lack of written documentation. In such a situation, the developer may even be interested in looking for examples on what kind of existing design rules to which she should adhere.

 Work has already been completed with regards to the first category, and while much research has been done in code suggestion, completely automated design rule generation seems largely unexplored (see Section~\ref{relatedWorks}). Therefore, an algorithm developed with the second category of developers in mind forms a natural bridge between human authored design rules and completely automated rule authoring. Consequently, we assume the developers for which we are designing our algorithm have some general familiarity with the code and can identify proper design rules if supplied with a set of rules to consider. Thus, feedback form the developer about hypothesized rules is dependable, enabling mixed human-AI rule authoring.


\subsection{Sources of Information} \label{infoSrcs}

Part of knowing which sources are most useful for deriving information about design rules is understanding where developers themselves look for information about a project and what kinds of information are useful to programmers.  

\subsection{Possible sources}

Several studies have demonstrated that as developers work and explore a project, they try to answer high-level questions about what, why, and how code causes different program states as they try to complete a task \cite{SadowskiEtAl2015, LaTozaMyers2010, LaTozaEtAl2007}. Researchers who investigated how programmers read and explore code have found that programmers' activities include checking implementation details, checking common style, and browsing. Programmers also performed code search to understand various dependencies within and between projects, why something is failing, and the side effects of a proposed changed \cite{SadowskiEtAl2015}. 

Ko et al investigate what sources of information developers use to accomplish a task and how that information affects developers' workflow throughout the day. They report that developers look at a variety of sources to accomplish a task, including check-in logs, bug reports, content management systems, version control systems, and other coworkers. Additionally, they identify knowledge about design and program behavior as the type of information that is most often deferred since unavailable coworkers are often the only source of knowledge. Developers sought out coworkers most often when there are information needs with regards to code design. Other questions to coworkers centered around following the team's conventions. Developers asked specific questions regarding what code caused different program states and behaviors and what statically related to the code of interest, but pursued the answers to these questions by starting with a hypothesis that they largely developed by using their intuition, asking coworkers, looking for execution logs, scouring bug reports, and using the debugger. They then often refined their hypothesis by using the search tool to answer questions about which sections of code performed similar operations \cite{KoEtAl2007}. 

Furthermore, Fritz et al conducted a survey of nineteen professional Java programmers in order to generate a degree of interest model to reflect developers' knowledge of code \cite{FritzEtAl2007}. The researchers' model was based on the total number of interactions the programmer has with an element and the recency of that interaction. Other factors they identified that could be used to improve their model included authorship of program elements, the role of elements, and the task being performed \cite{FritzEtAl2007}.

Based off of  these findings, we identified the debugger (?), code base, cursor location and search histories, and the developer herself as the most pertinent and accessible sources of information for our algorithm.

\textit{Debugger.} [Insert something later when we know what exactly we will be using about how we are using the debugger as a source of information ]

\textit{Cursor location.} [Insert something later when we know what exactly we will be using about how we are using the cursor location as a source of information ]

\textit{Search history.} [Insert something later when we know what exactly we will be using about how we are using the search history as a source of information ]

\textit{Developer.} [Insert something later when we know what exactly we will be using about how we are using the developer as a source of information ]



\subsection{Algorithm} \label{algm}

\subsubsection{Filtering information}
\subsubsection{Identifying token}
\subsubsection{Generating queries}
\subsubsection{Selecting queries to present to user}

\clearpage

%%%%%%%%%%%%%%%%%%%

\section{Tool interface} \label{toolInterface}
There are several major considerations even when we narrow our focus to a developer in the second situation. These considerations divide generally into considerations of what kinds of data are available and helpful for suggesting rules and how these kinds of data can be utilized to make relevant design rule suggestions.

There are sundry sources of data available that can and do provide information about the code base and what sections of the code in which the developer might be most interested. The most obvious source of information about the code is the code itself. Using a tool called srcML, the code can be put in a specific XML format that can then be searched using XPaths. Thus we can identify all kinds of information about the code itself, including function and class names and declarations, member variables, and relationships between classes like child and parent classes. Information is also available from the IDE itself. Such information can include the cursor's current and previous location, the user's search history, and which file windows are currently open in the project. Finally, the tool itself could be a possible source of information. For example, information about existing rules may be used to help improve the relevancy of the design rules presented to the developer.

There are a couple of approaches that could be taken with respect to different approaches to ensure relevant design rule suggestions. One way of viewing this challenge is to view it as a source of natural language programming (NLP) problem. One group of researchers were trying to improve code suggestions and viewed their challenge as analogous to a more traditional NLP problem of trying to fill in a missing word in a sentence. They developed a statistical language model to aid in providing suggestions for different code snippets \cite{RaychevEtAl2014}. We could think of our challenge as analogous to an NLP problem that aims to outline to a corpus of text or maybe, even more abstractly, find rhetorical devices in a text. Alternatively, we could simply rely more on traditional papers and research that have been conducted in this area that do not rely on NLP techniques to provide solutions. Both perspective may prove useful in the initial stages of research that entail exploring different ways that information is stored and what information about developer activity has seemed most helpful in designing related tools. 





\clearpage

%%%%%%%%%%%%%%%%%%%

\section{Related works}\label{relatedWorks}

Many tools and methods have been devised to augment programmers' ability to accomplish a task including code recommender systems for novices \cite{HartmannEtAl2010}, search tools for identifying bugs \cite{HovemeyerPugh2004}, search tools for identifying relevant API functions and the sections of code that need to be altered \cite{RongEtAl2016}, and code suggestion \cite{RaychevEtAl2014}. However, one area that seems largely unexplored is tools that facilitate the documentation of code.

Some researchers developed a tool to help a person who is new to an open-source software project find relevant artifacts to a project in order to facilitate their ability to more quickly perform a given task. They consider all of the previously produced material, including versions of the source, the bugs, archived electronic communication, and web documents, to suggest possibly relevant sources of information to a user \cite{CubranicMurphy2003}.

LaToza et al explored finding design patterns in an HTML document in order to perform code prediction \cite{LaTozaEtAl2019}. However, their tool does not identify design rules that might explain that pattern. 
ActiveDocumentation is a plug-in for IntelliJ IDEA that can be used to author rules and to view correct examples and violations of rules in a given code base in real time \cite{MehrpurEtAl2019}. The ActiveDocumentation plug-in helps the developer author, check, and fix rules, efficiently and in real time. It automatically updates its interface with correct examples and violations of design rules written in the program \cite{MehrpurEtAl2019}. Neither of these tools integrates information from sources outside of the code base such as cursor information or search history. Mylar is a tool designed for the Eclipse IDE that tries to highlight file names and code segments that are most likely to contain relevant information for a developer's current task. The highlighting is based off a model that depends on which elements of the code have been edited and accessed most recently \cite{KerstenMurphy2005}.




\clearpage

%%%%%%%%%%%%%%%%%%%
\section{Discussion}\label{disc}

It might be helpful to adapt our tool to be able to work with team environments. 

I don't really feel like discussing things any more. Why is this the only part that is called the discussion? I mean, what has been the rest of everything that has been written? Or did you just skip here? Is that why it has been called ``discussion"?

\clearpage
%%%%%%%%%%%%%%%%%%%

\section{Discussion}\label{disc}


\clearpage

%%%%%%%%%%%%%%%%%%%%
\section{References}\label{references}

\printbibliography

\end{document}
